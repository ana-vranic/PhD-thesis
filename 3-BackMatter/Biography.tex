\selectlanguage{english}

\normalsize

\chapter{Biography of the author}

Ana Vranic was born on November 23rd, 1993, in Cacak, Republic of Serbia, where she finished elementary and high school. In 2012 she enrolled BSc studies of Theoretical and Experimental Physics at the Faculty of Physics Belgrade and graduated in 2016 with a GPA of 9.24/10.00. In the same year, she started MSc studies at the Faculty of Physics and, after one year, finished them with a GPA of 10.00/10.00. Her master thesis, "Thermodynamics and electronic transport in Habard model on the triangular lattice", was done under Dr. Darko Tanaskovic in Scientific Computing Laboratory at the Institute of Physics Belgrade. During this research, she visited the institute Jozef Stefan in Ljubljana, for which she received the CEEPUS scholarship. Ana also won the "Prof. dr Ljubomir Ćirković" foundation award for best MSc thesis defended at the Faculty of Physics of the University of Belgrade.

In 2017, Ana Vranic started PhD studies at the Faculty of Physics in statistical physics. Under the supervision of Dr. Marija Mitrovic Dankulov at the Institute of Physics Belgrade. Since April 2018 Ana has been employed at the Institute of Physics Belgrade as a Research Assistant in the Scientific Computing Laboratory of the National Center of Excellence for the Study of Complex Systems. She participated in several projects:  the National Project ON171017 Modeling and Numerical Simulations of Complex Many-Body Systems, funded by the Ministry of Education, Science and Technological Development of the Republic of Serbia; Artificial Intelligence Theoretical Foundations for Advanced Spatio-Temporal Modelling of Data and Processes (ATLAS) project funded by the Science Fund of the Republic of Serbia and in REmote development of Autonomous Driving algorithms in a realistic environment (READ) project funded by Innovation Fund of Republic Serbia.

Ana Vranic has published four papers in peer-reviewed international journals. She also presented her work at international conferences through talks and poster presentations:

\begin{enumerate}
	\item \textbf{Vranić A}, Tomašević A, Alorić A, Mitrović Dankulov M. Sustainability of Stack Exchange Q$\&$A communities: the role of trust. EPJ Data Science. 2023 Feb 24;12(1):4.
	
	
	\item \textbf{Vranić A}, Smiljanić J, Mitrović Dankulov M. Universal growth of social groups: empirical analysis and modeling. Journal of Statistical Mechanics: Theory and Experiment. 2022 Dec 7;2022(12):123402.
	
	\item \textbf{Vranić A}, Mitrović Dankulov M. Growth signals determine the topology of evolving networks. Journal of Statistical Mechanics: Theory and Experiment. 2021 Jan 22;2021(1):013405.

	\item \textbf{Vranić A}, Vučičević J, Kokalj J, Skolimowski J, Žitko R, Mravlje J, Tanasković D. Charge transport in the Hubbard model at high temperatures: Triangular versus square lattice. Physical Review B. 2020 Sep 21;102(11):115142.
\end{enumerate}

\let\cleardoublepage\clearpage


%\pagebreak
%\selectlanguage{english}
%\thispagestyle{empty}
