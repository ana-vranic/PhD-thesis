
\chapter{The role of trust in knowledge based communities} % Main chapter title
\label{ChapterTrust}

Information and communications technologies (ICTs) have enabled faster and easier creation and sharing of knowledge. Furthermore, they have provided access to a large amount of data which enabled a detailed study of their emergence and evolution \cite{dankulov2015dynamics}, as well as user's roles \cite{saxena2021users}, patterns of their activity \cite{santos2019activity, slag2015one, chhabra2020activity}. 
However, relatively small attention was given to sustainability of SE communities. Most of the research was focused on the activity and factors that influence the increase of the users’ activity in these communities. Factors such as need for experts and the quality of their contributions have been thoroughly investigated \cite{dev2018size}. It was shown that growth of communities and mechanisms that drive it may depend on the topic around which the community was created \cite{santos2019self}. \\

\section{The Stack Exchange}

The Stack Exchange is a network of question-answer websites on diverse topics. In the beginning, the focus was on computer programming questions with StackOverflow \footnote{
	More information about StackOverlflow is available at: \url{https://stackoverflow.co/} and broad introduction to StackExchange network is available at: \url{https://stackexchange.com/tour}. 
}  community. Its popularity led to the creation of the Stack Exchange network that these days counts more than 100 communities on different topics. The SE communities are self-moderating, and the questions and answers can be voted, allowing users to earn Stack Exchange reputation and privileges on the site. 

The new site topics are proposed through site Area51 \footnote{Visit \url{https://area51.stackexchange.com/faq} for more details about closed and beta SE communities and the review process.}, and if the community finds them relevant, they are created. Every proposed  StackExchange site needs interested users to commit to the community and contribute by posting questions, answers and comments. After a successful private beta phase site reaches the public beta phase, other members are allowed to join the community. The site can be in the public beta phase for a long time until it meets specific SE evaluation criteria for graduation. Otherwise, it may be closed with a decline in users' activity. 

We focused analysis on four pairs of SE communities with the same topic. Astronomy, Literature and Economics are active communities \footnote{Astronomy, Literature and Economics graduated on December 2021 and during our research, they were still in the public beta phase.} The first time, these communities were unsuccessful and thus closed. We also compare closed Theoretical Physics with the Physics site, considering that those two topics engage similar type of users.

\subsection{Data}
Stack Exchange data are public and regularly released. As closed communities were active between 180 and 210 days, we extracted only first 180 days of data. Given that first few months can be crucial for further development of the community \cite{dover2020sustainable}, we are interested in early evolution of Stack Exchange sites. 

Detailed information about questions, answers, and comments are available for each SE community. Each post is labelled with a unique ID, the user's ID who made the post, and creation time. On Stack Exchange, users interact on several layers: Those interactions are considered positive.
\begin{itemize}
	\item posting an answer on the question; for every question, we extract IDs of its answers
	\item posting a comment on the question or answer; for every question and answer, we selected IDs of its comments
	\item accepting answer; for each question, we selected the accepted answer ID
\end{itemize}

Even though posts can be voted and downvoted, information about a user who voted is absent, so we do not consider these interactions between users. Comments can not be downvoted, while we find only around 3\% negatively voted answers and questions, Table \ref{tab:negint}.

\begin{table}[hbt!]
	\centering
	\caption{Percentage of negatively voted interactions}
	\label{tab:negint}
	\begin{tabular}{cc|cc}

		\hline
		Site                        & Status & Questions & Answers \\ \hline
		\multirow{2}{*}{Physics}    & Beta   & 5\%       & 4\%     \\
		& Closed & 1\%       & 2\%     \\ \hline
		\multirow{2}{*}{Astronomy}  & Beta   & 3\%       & 3\%     \\
		& Closed & 2\%       & 1\%     \\ \hline
		\multirow{2}{*}{Economics}  & Beta   & 4\%       & 4\%     \\
		& Closed & 7\%       & 4\%     \\ \hline
		\multirow{2}{*}{Literature} & Beta   & 2\%       & 5\%     \\ 
		& Closed & 2\%       & 1\%     \\ \hline \hline
		\textbf{Average}            &        & 3.2\%     & 3\%     \\ \hline 
	\end{tabular}
\end{table}


%The data contains information about the official StackExchange reputation of each user but only as a single value measuring the final reputation of the user on a day when data archive was released. Because of this significant shortcoming, we do not include this information in our analysis. In SE users can give positive or negative votes to questions and answers, and mark questions as favor, however the data is again provided as a final score recorded at the moment of the realise of the database. Since this does not allows us to analyse the evolution of scores, we omit this data from our analysis.\\


\subsection{Comparison between active and closed SE communities}

Table \ref{tab:site-info} compares the first 180 days between closed and active communities. When it comes to basic statistics, active communities had larger number of users, questions, answers and comments. Another simple indicator if community is going to graduate or decline can be time series of active questions for period of 7 days in Figure \ref{fig:active_questions}. The question is active if had at least one activity, posted answer ot comment during previous seven days. We find that live communities have larger number of active questions after first three months. Still, this difference is smaller for literature and astronomy. For astronomy we observe that closed community had higher number of active questions in the early period of community life. \\~\\


\begin{table}[h]
	\centering
	\caption{Community overview for first 180 days, Number of users $n_u$, number of questions $n_q$, number of answers $n_a$, number of comments $n_c$}
	\label{tab:site-info}
	\begin{tabular}{llccccc}
		\toprule
		Site                 & Status                           & First Date                     & $n_u$                    & $n_q$                & $n_a$                  & $n_c$ \\ \hline
		\multirow{2}{*}{Astronomy}  & \multicolumn{1}{l|}{Closed}      & \multicolumn{1}{c|}{09/22/10} & \multicolumn{1}{c|}{336}  & \multicolumn{1}{c|}{474}  & \multicolumn{1}{c|}{953}  & 1444     \\
		& \multicolumn{1}{l|}{Beta} & \multicolumn{1}{c|}{09/24/13} & \multicolumn{1}{c|}{405}  & \multicolumn{1}{c|}{644}  & \multicolumn{1}{c|}{959}  & 2170     \\ \hline
		\multirow{2}{*}{Economics}  & \multicolumn{1}{l|}{Closed}      & \multicolumn{1}{c|}{10/11/10} & \multicolumn{1}{c|}{275}  & \multicolumn{1}{c|}{368}  & \multicolumn{1}{c|}{458}  & 1253     \\
		& \multicolumn{1}{l|}{Beta} & \multicolumn{1}{c|}{11/18/14} & \multicolumn{1}{c|}{648}  & \multicolumn{1}{c|}{1024} & \multicolumn{1}{c|}{1410} & 3553     \\ \hline
		\multirow{2}{*}{Literature} & \multicolumn{1}{l|}{Closed}      & \multicolumn{1}{c|}{02/10/10} & \multicolumn{1}{c|}{284}  & \multicolumn{1}{c|}{318}  & \multicolumn{1}{c|}{523}  & 1097     \\
		& \multicolumn{1}{l|}{Beta} & \multicolumn{1}{c|}{01/18/17} & \multicolumn{1}{c|}{478}  & \multicolumn{1}{c|}{910}  & \multicolumn{1}{c|}{907}  & 3301     \\ \hline
		\multirow{2}{*}{Physics}    & \multicolumn{1}{l|}{Closed}      & \multicolumn{1}{c|}{09/14/11} & \multicolumn{1}{c|}{281}  & \multicolumn{1}{c|}{349}  & \multicolumn{1}{c|}{564}  & 2213     \\
		& \multicolumn{1}{l|}{Launched}    & \multicolumn{1}{c|}{08/24/10} & \multicolumn{1}{c|}{1176} & \multicolumn{1}{c|}{2124} & \multicolumn{1}{c|}{4802} & 15403    \\
		\bottomrule \\~\\
	\end{tabular}
\end{table}


\begin{figure}
	\centering
	\includegraphics[width=\linewidth]{figures/stackexchange/active_questions.pdf}
	\caption{Number of active questions within 7 days sliding windows. Solid line - active sites; dashed lines - closed sites.}
	\label{fig:active_questions}
\end{figure}

Similarly, the official Stack Exchange community evaluation process considers simple metrics \footnote{\href{https://stackoverflow.blog/2011/07/27/does-this-site-have-a-chance-of-succeeding/}{https://stackoverflow.blog/2011/07/27/does-this-site-have-a-chance-of-succeeding/}}. To determine the the success of sites  they measure how many questions are answered, how many questions are posted per day, and how many answers are posted per question.  There are two measures: the number of avid users and the number of visits that are not easily interpreted from the data. The site is \textit{healthy} if it has 10 questions per day, 2.5 answers per question and more than $90\%$ of answered questions. For less than $80\%$ of answered questions, 5 questions per day and 1 question per answer site \textit{needs some work}. 

We calculated Stack Exchange statistics for astronomy, economics, literature and physics and results are presented in the Table \ref{tab:se_c}. After observed period of 180 days only live physics is healthy site while other live communities are at least in two criteria labeled as \textit{okay}. Closed sites mostly \textit{need some work}, the exception is closed astronomy. For example it has \textit{excellent} percent of answered questions and \textit{okay} answer ratio.  

%While Physics community was clearly more successful than Theoretical Physics and other considered communities, we see that these differences are not as clear if we compare three other pairs of communities. For instance, some of the parameters for closed Astronomy community were better than for the community that is still alive. Similar results were found for Economics and Literature. 

\begin{table}[h!]
	%\footnotesize\sf\centering
	\caption{Community overview for first 180 days according to SE criteria  }
	
	\label{tab:se_c}
	\begin{tabular}{ccccc}
		\hline
		
		Site & Status &  Answered & Questions per day & Answer ratio \\ \hline
		\multirow{2}{*}{Astronomy} & \multicolumn{1}{c|}{Closed} & \multicolumn{1}{c|}{\textbf{95} \%}  & \multicolumn{1}{c|}{2.62} & \multicolumn{1}{c}{\underline{2.02}} \\
		& \multicolumn{1}{c|}{Beta} & \multicolumn{1}{c|}{\textbf{96} \%}  & \multicolumn{1}{c|}{3.57} & \multicolumn{1}{c}{\underline{1.49}} \\ \hline
		\multirow{2}{*}{Economics} & \multicolumn{1}{c|}{Closed} & \multicolumn{1}{c|}{68 \%}  & \multicolumn{1}{c|}{2.04} & \multicolumn{1}{c}{\underline{1.25}} \\
		& \multicolumn{1}{c|}{Beta} & \multicolumn{1}{c|}{\underline{84} \%}  & \multicolumn{1}{c|}{\underline{5.66}} & \multicolumn{1}{c}{\underline{1.37}} \\ \hline
		\multirow{2}{*}{Literature} & \multicolumn{1}{c|}{Closed} & \multicolumn{1}{c|}{79 \%}  & \multicolumn{1}{c|}{1.77} & \multicolumn{1}{c}{\underline{1.65}} \\
		& \multicolumn{1}{c|}{Beta} & \multicolumn{1}{c|}{74 \%}  & \multicolumn{1}{c|}{\underline{5.04}} & \multicolumn{1}{c}{\underline{1.10}} \\ \hline
		\multirow{2}{*}{Physics} & \multicolumn{1}{c|}{Closed} & \multicolumn{1}{c|}{83 \%}  & \multicolumn{1}{c|}{1.93} & \multicolumn{1}{c}{\underline{1.64}} \\
		& \multicolumn{1}{c|}{Beta} & \multicolumn{1}{c|}{\textbf{93} \%}  & \multicolumn{1}{c|}{\textbf{11.76}} & \multicolumn{1}{c}{ \textbf{2.74}} \\ \hline \hline
		{Stack Exchange criteria} & excellent & $>$ 90 \% & $>$10 & $>$ 2.5   \\
		& needs some work & $<$ 80 \% & $<5$ & $<$ 1   \\ \hline
		
		
	\end{tabular}
	
\end{table}

This simple measurements presented in tables \ref{tab:site-info} and \ref{tab:se_c} and in figure \ref{fig:active_questions} do not provide us clear indications about community sustainability. Only for physics topic the difference between active and closed community is evident, while for other communities it is not so clear. Thus, we need deeper insights into structure and dynamics of these communities to understand. The structure of social interactions within communities and dynamics of collective trust may provide better explanation why some communities succeed and other died. 






       
