

\section{Models}

\subsection{The random graph model }

The random graph moddel was introduced by Erdos and Renyi in 1959. This model has $N$ disconnected nodes. With probability $p$ each pair of nodes can be connected.  The network is characterized only by number od nodes and links, $G(n, p)$. 

As this process is stochastic, the network with same parameters n and p, does not have to be same structure, so it is necessary to consider the ensemble of networks. Then the mean number of links depends on the model parameters:
$\langle m \rangle = n(n-1)p / 2$. The expected value of node degree can be predicted as $\langle k \rangle = (n-1)p$.
The probability $p$ is defined as density of the network. 
The probability $p(k)$ follows the binomial distribution of the form:

$p(k) = p^k(1-p)^{n-1-k}$. For large values of $n$, this becomes $p(k) = e^{-k}k^{-k} / k!$, which is Poisson distribution. 

In the case of a large random networks, the average path length is given as $l = \frac{ln n - \gamma}{ln(pn)} + \frac{1}{2}$. This means that random graph has a very small average path length. This is characteristic of many large networks. 

The clustering coefficient $C=p$, so for sparse ER graphs the clustering is very small, much smaller than in real world networks.

Increasing the probability $p$, the giant component may appear. This is sub-graph whose size is proportional to the size of the network. Such change in the network is phase-transition and it is important as small change in the probability p leads to fundamental change in the system properties. There are two limits of this model, when $p=0$, network is disconnected. If $p=1$, then network is fully connected, and giant component is with size $O(1)$.  This phenomena is related to percolation phase transition. On the threshold $p_c$, the component whose size is proportional to $n^{2/3}$ emerges. Average path length between two nodes, at critical point is proportional to the $ln N$. The small, logarithmic distance is the origin of the "small-world" phenomena. 

The interesting behavior of this model is that, with increasing p nodes tend to organise in giant component. The subcritical , $k <1$ where all components are dimple and small. The size of larges component is $s=O(ln n)$. In critical regime $k=1$, the size of largest component is $s=O(n^{2/3})$. Supercritical $k>1$, where the probability of having giant component is 1.  


\subsection{Small world networks}

In the 1999. Watts and Strogatz introduces "small-world" model. This model can generate the networks with small diameter and high clustering coefficient. Their idea is to start from grid like network, where all nodes have same number of neighbors, like ring-lattice or hexagonal lattice, where each node is connected to k nearest neighbors. Such network has high clustering coefficient, as any pair of consecutive neighbors are connected forming a triangles, while in contrast the network has high average shortest path, as nodes on the oposite sides of the lattice are not connected. The goal of this model is to connect distant nodes and reduce the average path lenght in the network. This can be simply done by randomly rewiring nodes in the network, with probability $p$. Model interpolates between regular network $p=0$ and random graph $p=1$, and for some critical probability we can achieve small world networks. 

%TODO
% mozda dodati sliku ovog moddela

The average shortest-path length from the model is close to that of an equivalent network, and much lower than that of the lattice. The clustering coefficent from the model is still close to that one in the lattice and much larger than in random network. 

The degree distribution of this model obviously is not power-law. In regular network, all nodes have equal degree, while in random networks degree distribution becomes Poisson. 

\subsection{Barabasi-Albert model}

The random network model differs from real networks in the two characteristics, growth and preferential attachment. In static models, number of nodes is fixed, while in growing models we try to simulate the continuous change in the system. More important ingredient, are linking rules. In real networks, new nodes tend to link to more connected nodes.

This model is defined as follows, we start from $m_0$ nodes, randomly connected, and at each timestep we add new node with m links that will connect to $m$ nodes already present in the network.  The probability that new node connects to node $i$ depends on node degree $k_i$ as

\begin{equation}
P(k_i) = \frac{k_i}{\sum_jk_j} 
\end{equation} 

New node can connect to any node in the network, however nodes with larger degree have higher probability to link new nodes. After time $t$ the model generates network with $N=t+m_0$ nodes and $m_0+mt$ links. Degree distribution is power-law with exponent $\gamma=3$. As network grows nodes with larger degree becomes bigger, so we end up with few nodes with many links, called hubs. Two simple mechanisms are responsible for emergence of scale-free networks. 

 \textit{degree distribution}
 
To understand the emergence of scale-free properties we need to analyze the evolution of degree distribution. The rate at which an existing node get new links as result of new nodes connecting to it is

\begin{equation}
\frac{dk_i}{dt} = mP(k_i) = m\frac{k_i}{\sum_jk_j}
\end{equation}

each new node arrives with m links. The sum is $2mt - m$ so the equation for large t becomes:

\begin{equation}
\frac{dk_i}{k_i} = \frac{1}{2}\frac{dt}{t}
\end{equation}

solving this equation we get that degree of node in time step t is $k_i(t)=m(\frac{t}{t_i})^\beta$, where $\beta=1/2$. 

We note that degree of each node increase following power-law; the growth in degrees is sub linear, as each new node has more nodes to link than previous. The eirlier node $i$, the higher is its degree. Hubs are large as they arrived early in the network. 

In summary, the analytical calculations predict that the Barabási-Al-
bert model generates a scale-free network with degree exponent 3. The degree exponent is independent of the m and m 0 parameters. The degree distribution is stationary explaining how different systems have similar structural properties. 

In summary, the absence of preferential attachment leads to a growing network with a stationary but exponential degree distribution. In contrast the absence of growth leads to the loss of stationarity, forcing the network to converge to a complete graph. This failure of Models A and B to reproduce the empirically observed scale-free distribution indicates that growth and
preferential attachment are simultaneously needed for the emergence of the scale-free property.

In the past decade we witnessed the emergence of two philosophically different answers. The first one views preferential attachment as the interplay between random events and some structural property in the network. The second assumes that each new node or link balances conflicting needs. 

The BA model postulates the presence of preferential attachment. Yet, we can build models that generate scale free networks without preferential attachment. The link selection model offers the simplest mechanism that generates a scale-free network. At each time step we add new nodes to the network, we select link at random and connect the new node to one of the two nodes at the end. The higher is degree of the node, the higher is chance that node is located at the end of chosen link. The more k-degree nodes are there, the more likely is that k node is at the end of chosen link. Probability that node at the end of randomly choosen link has degree k is $q_k = Ckp_k$. The fact that bias is linear with k indicates that the link selection model builds scale-free networks. 
Copying model can also generate scale-free networks. In each time step a new node is added to the network. To decide where it connects we randomly select node u. Then with probability $p$ new node links to $u$, otherwise with probability $1-p$ we randomly choose an outgoing link of node $u$ and link the new node to its target. The likelihood that new node connects to degree-k node is $P(k)=\frac{p}{N} + \frac{1-p}{2L}k$, the second part is equivalent in selecting a node to randomly selected link. The popularity of the copying model lies in its relevance in real systems. It is common in social networks, citation networks or even protein interactions. 
in optimization, when new nodes balance conflicting criteria as they decide where to connect

\textit{diameter}
The network diameter, represents the maximum distance in the BA model, $d \sim \frac{lnN}{lnlnN}$. The diameter grows slower than $lnN$, making the distances in BA model smaller than in random graph. The difference is found for large N. 
\textit{clustering}
The clustering coefficient of the BA model follows $C \sim \frac{ln N^2}{N}$. It is different from clustering found in random networks, and BA networks are in general more clustered. 

\subsection{Nonlinear BA model}

In summary, nonlinear preferential attachment changes the degree
distribution, either limiting the size of the hubs $(\alpha < 1)$, or leading to su-
per-hubs ($\alpha > 1$, ). Consequently, $P(k)$ needs to depend strictly lin-
early on the degrees for the resulting network to have a pure power law p k .
While in many systems we do observe such a linear dependence, in others,
like the scientific collaboration network and the actor network, preferen-
tial attachment is sublinear. This nonlinear $P(k)$ is one reason the degree
distribution of real networks deviates from a pure power-law. Hence for
systems with sublinear  the stretched exponential (5.23) should offer a
better fit to the degree distribution.


In real systems preferential attachment can be more influenced by the age of the node. If parameter alpha is negative, ageing effect overcomes the role of preferential attachment, and scale-free properties are lost. For large negative alpha, the network turns into the chain, where the youngest nodes are the most attractive. On the other hand for a positive alpha, new nodes will link to older nodes. Positive alpha makes the network more heterogeneous, and scale-free nature still exist but exponent gamma is different from 3.  for the high alpha all nodes will tend to connect to oldest node. 

In the general ageing model, we have linking rules where rules connecting probability depends on both of node degree and age difference between new and old node.  With parameters alpha and beta we can control the structure of generated networks. I already talked about some limits of the general model. We saw that for specific parameters there are SF networks, BA model, if we move from that point SF behaviour with power-law with exponent 3 is lost. And other classes of networks can appear. In general, model, when alpha and beta are both positive, rich get richer phenomena is more promoted. On the other hand, the region where beta is positive and alpha negative can be interesting, because SF networks can appear only along the critical line. 

In growing network models is considered that at each time step one node is added to the network. The remaining question is if there is any change if network growth is not linear anymore and how does it influence the structure of obtained networks.  In this work, we use numerical simulations to explore the case when $M(t)$ is a correlated time-varying function and study how these properties influence the structure of generated networks for different values of parameter $-\infty<\alpha\leq-1$ and $\beta\geq1$ and constant $L$.

%\section{Graph isomorphism}

%Weisfeiler-Lehman Test

\section{Distributions}

\subsection{power-law}
Power-law distributions characterize many social and biological systems. Power-law distributions are also easy to generate. 

the distributions: basic definitions and properties

The nonnegative random variable X is said to have a power law distribution if 
\begin{equation}
Pr[X>x] \sim c x ^{-\alpha}
\end{equation}

for constants $c>0$ and $\alpha >0$. In power-law distribution asymptotically the tails fall according to power $\alpha$. Such distribution leads to much havier tails than other common models, as exponential distribution. 

One specific power-law distribution is Pareto distribution which satisfies 

\begin{equation}
	Pr[X>x] = \frac{x}{k} ^{-\alpha}
\end{equation}

for some $\alpha>0$ and $k>0$. The Pareto requires $X>k$. The density function of Pareto distribution is $f(x)=\alpha k^\alpha x^{-\alpha-1}$. For power law distribution $\alpha$ is in the range $0 < \alpha < 2$, in which case X has infinite variance, if alpha <1 X has infinite mean.

if X has power law distribution, then a in log-log plot $Pr(x)$ will behave as straight line. For the specific case of Pareto distribution, the behaviour is exactly linear as $ln (Pr(x)) = -\alpha (lnx -ln k) $. Similarly the density function is also straight line. $ln(f(x)) = (-\alpha - 1) ln (x) + \alpha ln (k) + ln(a)$

\subsection{log-normal}
Many measurments in the nature show a more or less skewed distribution. They are common when mean values are low, variances are large and values can not be negative as example in distribution of minearal resources in the Eart. Such skewed distributions often closely fit to log-normal distribution. 

What is the difference between normal and lognormal variability? A major difference is that effect can be additive or multiplicative, leading to normal or lognormal distribution. Basic principles of additive and multiplicative effects can be easily demonstrated with the help od two dices. Adding the two numbers, which is the principle of the most games, leads to values from 2 to 12 with mean of 7, and simetrical frequency distribution. Multiplying the two numbers, leads to values from 1 to 36 with highly skewed distribution. Although these examples are not normal or lognormal they give us clear difference how different distributions can emerge. 

Log-normal distributions are usually characterized in the term of the log-transformed variable, using as parameters the expected value, or the mean, and the standard deviation. This characerization can be advantageous as by definition log-normal distributions are simetrical at the log level. 

The basic properties of the lognormal distributions

Random variable $X$ if $log(X)$ is normally distributed., if $Y=ln X$ has normal Gaussian distribution.  

Only positive values are possible for the variable and distribution is skewed. Two parameters are needed to specify lognormal distribution. Traditionaly the mean $\mu$ and standard deviation $\sigma$ or the variance of the $\sigma^2$ of $log(X)$ are used. However there are clear advantages of using transformed data, $\mu^{*} = e^{\mu}$, $\sigma^{*}= e ^{\sigma}$. The median of this lognormal distribution is $med(X)=\mu^{*}=e^{\mu}$, since $\mu$ is median of the $log(X)$.

\begin{equation}
f(x) = \frac{1}{x \sigma \sqrt{2\pi}}exp(-\frac{1}{2\sigma^2}(log(x)-\mu)^2)
\end{equation}

The mean is $exp(\mu + \sigma /2)$ and variance is $(exp(\sigma^2)-1)exp(2\mu+\sigma^2)$. 

Estimation: The asymptotically most efficient (maximum likelihood) estimators are 
\begin{equation}
x* = exp (\frac{1}{n}\sum_{i=1}^n log(x_i)) = (\prod_{i=1}^nx_i)^{\frac{1}{n}}
\end{equation}

\begin{equation}
s* = exp([\frac{1}{n-1}\sum[log(\frac{x_i}{x*})]^2]^{\frac{1}{2}})
\end{equation}


The lognormal distribution is skewed with mean $e^{\mu + \frac{1}{2}\sigma^2}$, median $e^\mu$ and mode $e^{\mu - \sigma^2 }$. It has finite mean and variance, in contrast to the power-law distribution.  

Despite it has finite moments, the lognormal distribution can be similar to power-law. If $X$ has a lognormal distribution then loglog plot of density function can apear as straight line for a large portion of a body of distribution. If the variance is large, the distribution may appear linear on log-log plot for several orders of magnitude. The variance of the corresponding normal distribution is large, the distribution may appear linear on a log-log plot. To see this we can check the logarithm of density function. 
If $\sigma$ is large then the quadratic term will be small for large range of x values, so the logarithm of the density function will appear almost linear for large range of values. 
Recall that normal distribution have property that the sum of two independent normal variables is normal variable. It follows that product of two lognormaly distributed random variables also has a lognormal distribution. 

%\section{Scale-free networks}

%The study of scale ivariance has a long tradition. Among the fields where this property was analysed were the theory of critical phenomena, percolations and fractal geometry. One of the first examples considered eas the price fluctuations  of cotton in commodities market (Mandelbort, 1963). The future price can not be obtained with arbitary precision from past series., still this series have some form of regularity. The curves for daily, weakly and montly price fluctuations are statistically similar. The fact that some features are found at different time scales is typical sign of fractal behaviour. Similarly in the case of coastline lenght we find fractal behavior. If we try to measure the total lenght, the real shape is so complicated that we always miss some part. 

%Fractal behaviour might refer to different properties. In some systems scale-free structure is in shape. In this class the fractal shape can be robust, as in the case of branched patterns or electric breakdown. We say robust because these phenomena happen for varaity of external conditions. In the same class we have other systems that are more fragile, in the sense that they arise after precise tuning of some physical quantity. This is in the case of percolations and critical phenomena. Scale-free invariance may be related with dynamics or evolution of the system. The time activity of the system may display self-similar behaviour. The only sign of fractal behaviour is the mathematical form, power-law fluctuations of the time-series. 

%The self-similarity can be present in the way the different parts of a system interact with each other. This is the case with self-similar graphs and the power-law scaling appers in the distribution of topological quantities like the number of interactions per part  of the system. These phenomena are fractals in the topology. 

%\section{Scale-invariance and power laws}

%The mathematical form of self similarity is represented by powe-laws. Whenever the function $y = f(x)$ can be represented as a constant to the power of x. The physical example is elastic force, the gravitational and electrostatic force. In the case of fractals, their geometry can be identified by considering the numbe of boxes $N(\epsilon)$, of linear size $1/\epsilon$. $N(\epsilon) = \epsilon ^{-D}$, where D is called fractal dimension. D can also be defined using mass relation $M = L^ D$.

%Scale invariance is not restrected to geometry, but also appears in dynamical systems. In this case we have power-law distribution for different physical\setsecnumdepth{subsection} quantit\setsecnumdepth{subsection}ies. For example the evolution of some systems (sanspiles, number of species in the ecosystem) proceeds  with series of causally connected events whose size s is distributed as power-law. $P(s) s^{-\tau}$.

%\subsection{plotting a power-law}

%if we plot the data on double logarithmic scale, we should obtain a straight line. $y = x^{\alpha}$, $ln(y) = \alpha ln(x)$. The tail of the distributon is very noisy, and it is general feature of many experiments. To avoid the fluctuations in the tail it is common to use "binning" or cumulative distribution. 

%In this method the noise reduction is done by deviding the $x$ axis into bins, and averaging the data within each bin. As an example we can take the frequency of numbers between numbers $1$ and $10$. Let assume that bin is $10$ units wide, we can represent all data in one point $b$, with $xb$ given by average of the bin extremes and $yb$ given by the avarage value of $ys$. If the bin size is constant for large values of $x$ the density of bins becomes very large as well. 

%In the case of powerlaws it is usufull to use logarithmic binning. For example take the size of the first bin to be 2, and the others are power of 2. $2^1, 2^2...$. A possible choice is to take as yb the average of the values ys in the bin, and as xb the geometrical average of the bin extremes. Bins are equaly spaced on logarithmic scale. The base can be any number larger than 1, for example 1.2. The drawbacks of this procedure are following: this method does not complitly reduce the noise and the choice of the most appropiate bin size must be determined by using trial and error. In general, small and noisy dataset , the behavior in the tail of distribution can be lost if the bins are too wide. Too small bins will not average the present fluctuations

%Using the method of cumulative distribution instead of calculating the probability that certain value x appears in the experiment, we focus on the probability that an outcome is larger than x. 

%For power-law, 
%$P>(x) = \int_{x}^{\inf} P(x^{'})dx^{'}$.
% For power law cumulative distribution is still powerlaw but with modified exponent. $-\alpha+1$. Still, in this method, if exponent is close to one, integral does not behave like powe-law, reather logarithm. The upper limit is a finite value $x_{max}$ and this can change the shape of curve. 

%The cumulative distribution will resemble a power law only as the value of $xmax$ tends to infinity. Otherwise, the deviation from the straight line could make estimating the exponent very difficult.

%\subsection{multiplicative processes}

%While many power laws are originated by some ‘complex’ mechanism, some others have a very simple explanation. By using multiplicative processes we can obtain quite naturally both power laws and log-normal distributions (that can look like power laws) (Goldstein, Morris, and Yen, 2004). We do not enter here into the debate over whether observed data can be best fit by power law or log-normal variables. Here it is enough to note that the mechanism of multiplicative process is probably the most immediate model for fat-tail phenomena in nature since it naturally produces both.

%Many textbooks and scientific papers deal with this topic. Some of them are very beautiful and complete and we suggest them for further reading (e.g. Mitzenmacher 2004; Newman 2005). Suppose you have an evolution process, where for example an organism transforms itself in time. As a general statement, the state S t at time t will be determined by the previous states and by the external conditions. Whenever the state of the system can be written as we have a multiplicative process. In other words, in a multiplicative process, the state of the system at time t is proportional to the state at time t-1. In biology this could represent the fact that the growth of an organism is ruled by its body mass at the previous step. In the case of city growth (Zanette and Manrubia, 1997; Manrubia and Zanette, 1998; Gabaix, 1999) this equation states that the population at a certain time step is proportional to what it was previously. In both cases the proportionality constant is given by the factor t that can change its value at any time step. Turning back to eqn 4.114 we can immediately see that the variable S t is determined by the product of the various tau where tau is between 0 and t This sum of the logarithms of the tau (under very mild conditions) is a variable following a normal distribution (regardless of the distribution of the tau ). This result comes from the application of the ‘central limit theorem’. This theorem states that, in certain very general hypotheses, the sum of identi- cally distributed random variables with finite variance is a new stochastic variable normally distributed. Therefore, if ln(S t ) is normally distributed, the variable S t is log-normally distributed. This very simple mechanism has been rediscovered and explained over and over many times since the definition of log-normal distributions in 1879 (McAlister, 1879). One of the first applications of these ideas traces back at least to the economist Gibrat (1930, 1931) who uses essentially this model under the name of proportionate effect. Using a different terminology, a somewhat similar idea was introduced at the beginning of last century for biological problems (Kapteyn 1903, 1918). With this idea we have two possible outcomes. As explained above we have true log-normal distributions that can easily be confused with power laws. This happens whenever the log-normal distribution is studied in a range of k for which $\sigma >> ln(k)$. On the other hand, a very similar situ- ation also triggers the formation of true power laws as shown in the next subsection. Powerlaws from multiplicative processes.

%\section {Preferential attachment}

%One of the most successful applications of multiplicative processes is given
%by preferential attachment. To date, this is the most successful mechanism adopted in the study of growing networks. Interestingly, the idea that we are going to explain has been independently rediscovered several times in different fields and ages. Precisely for this reason it has also been given several names. For example: Yule Process, Matthew effect, Rich gets richer, Preferential Attachment, Cumulative advantage. In the community there is some agreement (Mitzenmacher, 2004; New- man, 2005) that the first to present this idea has been G. Yule (1925) in order to explain the relative abundance of species and genera in biological taxonomic trees. As shown in Chapter 8 when considering a set of biolog- ical species we have that the classification (taxonomic) tree has scale-free properties. The null hypothesis consists in considering that the set of species arises from a common evolution. Therefore we consider one parent species and after mutation we obtain a new one that very likely can be grouped in the same genus. Every now and then though, speciated species (the new ones) can be so different from the parent one that they can form a new genus on their own (or be grouped in an existing different one). The probability of speciating will be larger for genera that are already large, since mutation rate is constant for any individual.

%This explanation allow us to focus on the two ingredients of the model. Firstly you have to invoke a certain a priori dynamics (hereafter called growth). Secondly, this dynamics selects successful elements and makes them even more successful (hereafter called preferential attachment). In detail, take a set of elements each of which is characterized by a certain number N i t . As a possible example this could be the number of different genera that have i species per genera. The set can also be a set of vertices in a graph and the number N i t can represent the number of pages whose in-degree is i. Now let us introduce a rule that introduces new elements in the set; these elements will not be shared equally between the older ones, but rather will be assigned more to those that already have many. Let us consider that N i t gives the number of vertices with certain degree i (the total

\section{Multifractality of the signals}

Multifractal detrended fluctuation analysis (MFDFA) \cite{kantelhardt2002, ihlen2012} to estimate multifractal Hurst exponent H(q). For given time series $\{x_i\}$ with length N, first we define global profile in the form of cumulative sum, equation \ref{eq:cumsum}, where where $\langle x\rangle $ represents average of the time series:
\begin{equation}
Y(j) = \sum_{i=0} ^j (x_i - \langle x\rangle), \quad j=1, ..., N
\label{eq:cumsum}
\end{equation}

Subtracting the mean of the time series is supposed to eliminate global trends. The profile of the signal Y is divided into $N_s = int (N/s)$ non overlapping segments of length s. If $N$ is not divisible with s the last segment will be shorter. This is handled by doing the same division from the opposite side of time series which gives us $2N_s$ segments. From each segment $\nu$, local trend $p^m_{\nu, s}$ - polynomial of order m - should be eliminated, and the variance $F^2(\nu, s)$ of detrended signal is calculated as in equation \ref{eq:var}:
\begin{equation}
F^2(\nu, s) = \frac{1}{s}\sum_{j=1}^s \left[Y(j) - p^m_{\nu, s}(j)\right]^2
\label{eq:var}
\end{equation}
Then the q-th order fluctuating function is: 
\begin{equation}
F_q(s) = \left\{\frac{1}{2N_s}\sum_{\nu}^{2N_s}\left[F^2(\nu, s)\right]^{\frac{q}{2}}\right\}^{\frac{1}{q}},  q \neq 0 \nonumber
\end{equation}
\begin{equation}
F_0(s) = \exp \left\{\frac{1}{4N_s}\sum_{\nu}^{2N_s}ln \left[F^2(\nu, s)\right]\right\}, q=0
\end{equation}

The fluctuating function scales as power-law $F_q(s) \sim s^{H(q)}$ and the analysis of log-log plots $F_q(s)$ gives us an estimate of multifractal Hurst exponent $H(q)$. Multifractal signal has different scaling properties over scales while monofractal is independent of the scale, i.e., H(q) is constant. 

\section{Dynamical reputation model}

Any dynamical trust or reputation model has to take into account distinct social and psychological attributes of these phenomena in order to estimate the value of any given trust metric \cite{duma2005dynamic}. First of all, the dynamics of trust is asymmetric, meaning that trust is easier to lose than to gain. As part of asymmetric dynamics, in order to make trust easier to loose the trust metric has to be sensitive to new experiences (recent activity or the absence of the activity of the agent), while still maintaining nontrivial influence of old behavior. The impact of new experiences has to be independent of
the total number of recorded or accumulated past interactions, making high levels of trust easy to lose. 
Finally, the trust metric has to detect and penalize both the sudden misbehavior and the possibly long term oscillatory behavior which deviates from community norms.

We estimate dynamic reputation of the Stack Exchange users using Dynamic Interaction Based Reputation Model (DIBRM) \cite{melnikovDynamicInteractionBasedReputation2018}. This model is based on the idea of dynamic reputation, which means that the reputation of users within the community changes continuously through time: it should rapidly decrease when there is no registered activity from the specific user in the community (reputation decay), and it should grow when frequent, constant interactions
and contributions to the community are detected. The highest growth of user's reputation is found through bursts of activity followed by short period of inactivity. 

In our implementation of the model, we do not distinguish between positive and negative interactions in the Stack Exchange communities. Therefore, we treat any interaction in the community (question, answer or comment) as potentially valuable contribution. In fact, evaluation criteria for Stack Exchange websites going through beta testing, described in SI, do not distinguish between positive and negative interactions.
The percentage of negative interactions in the communities we investigated was below 5\%, see Table 1 in SI. Filtering positive interactions would also require filtering out comments because they are not rated by the community, and that would eliminate a large portion of
direct interactions between the users of a community, which is essential for estimating their reputation.

In DIBRM, reputation value for each user of the community is estimated combining three different factors: 1) \textit{reputation growth} - the cumulative factor which represents the importance of users' activities; 2) \textit{reputation decay} - the forgetting factor which represents the continuous decrease of reputation due to inactivity; \textit{the activity period factor} - measuring the length of the period of time in which the change of reputation happened. In case of Stack Exchange communities, the forgetting factor has a literal meaning, as we can assume that past contributions provided by a user are being forgotten by active users as their attention is captured by more recent content.

In line with the the basic dichotomy of reputation dynamics, which revolves around the varying influence of past and recent behavior, DIBRM has two components: \textit{cumulative factor} - estimating the contribution of the most recent activities to the overall reputation of the user; \textit{forgetting factor} - estimating the weight of past behavior. Estimating the value of recent behavior starts with the definition of the parameter storing the basic value of a single interaction $I_{b_{n}}$. Cumulative factor $I_{c_{n}}$ then captures the additive effect of recent successive interactions. The reputational contribution $I_n$ of most recent interaction $n$ of any given user is estimated in the following way:

\begin{equation}\label{eq:ibn}
I_n = I_{b_{n}} + I_{c_{n}} = I_{b_{n}} (1+  \alpha  (1-\frac{1}{A_{n}+1}))
\end{equation}

Here, $\alpha$ is the weight of the cumulative part and $A_{n}$ is the number of sequential activities. If there is no interaction at $t_n$, this part of interactions has a value of 0. Important property of this component of dynamic reputation is the notion of sequential activities. Two successive interactions made by a user are considered sequential if the time between those two activities is less or equal to the time parameter $t_{a}$ which represents the time window of interaction. This time window represents maximum time spent by the user to make a meaningful contribution (post a question or answer or leave a comment).

\begin{equation}\label{eq:deltan}
\Delta_{n}=\frac{t_{n}-t_{n-1}}{t_{a}}
\end{equation}

If $\Delta_{n} < 1$ is less than one the number of sequential activities $A_{n}$ will increase by one, which means that the user is continuing to communicate frequently. On the other hand, large values $\Delta_{n}$ greatly increase the effect of the forgetting factor. This factor plays a major role in updating the total dynamic reputation of a user in each time step (after every recorded interaction):

\begin{equation}\label{eq:tn}
T_{n}=T_{n-1} \beta^{\Delta_{n}}+I_{n}
\end{equation}

Here, $\beta$ is the forgetting factor. In our implementation of the model, the trust is updated each day for every user irrespective of their activity status. Therefore, the decay itself is a combination of $\beta$ and $\Delta_n$: the more days pass without recorded interaction from a specific user, the more their reputation decays. Lower values of beta lead to faster decay of trust as shown on figure \ref{fig:paper_summary}.