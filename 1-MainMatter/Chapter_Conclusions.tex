\chapter{Conclusions} % Main chapter title
\label{Ch:Conclussion}

%%TODO: Uvodni pasus
In this thesis, we studied the complex network models to understand the evolution of online social systems. %Considering the properties of the real communities, the evolving network models could be improved and more realistic representations. 
The complex systems change over time, even though we often find the system's collective behaviour that stays universal. The specific interactions among elements could lead to different kinds of organizational patterns. The goal of this thesis tries to understand factors that drive the system's growth, change its structural properties and their sustainability.
%contribute to the growth of the systems, structure properties and their sustainability.
%%TODO factors that drive the growth of the system, change of its structure properties and their sustainability.
The underlying methodology is introduced in chapter \ref{Ch:Method}. The first part explained the most important properties of network structure and the growing network models. The second part describes the statistical methods useful for the empirical analysis of the properties of the complex system.  

In chapter \ref{Ch:signals}, we discussed how nonlinear growth signal shapes the structure of the complex network. 
%The previous models combined linking rules with constant growth, but we added one more parameter, the fluctuating growth signals, in this research. 
%%TODO Ovu recenicu bih malo drugacije: 
The previous models combined linking rules with constant growth; however, empirical analysis of various real systems and agent-based simulation \cite{mitrovic2012, mitrovic2015} have indicated that properties of growth signal influence the dynamics of complex systems, as well as the structure of its interaction network. To investigate the connection between the features of the growth signal and the structure of an evolving network, we added one more parameter in the growth of the aging network model, the fluctuating growth signal, and examined how network properties change with the signal features.
The most considerable influence is found on scale-free networks. Many interaction networks from social, technological or biological systems have scale-free structure; they are correlated and clustered. These results suggested that it is important to study growing signals' properties. Signals from natural systems show trends and cycles and are characterized by long-range temporal correlations. The structure of the generated complex networks depends on the signal properties, and it is necessary to quantify these properties as they affect the network's topology differently. For example, the most significant difference between networks generated with fluctuating and constant signals is found for signals with multi-fractal properties. This difference is more negligible for monofractal signals or uncorrelated white noise. Fluctuating signals promote the creation of hubs in the network and shorten the paths between nodes.

Chapter \ref{Ch:Groups} presented the results of the universal characteristics of the growth of online social groups—the growth of the system influence the structure of the interaction network. The distribution of the sizes of the complex systems usually follows some universal curve. In many cases, it is lognormal or power-law. The distribution of the dimensions of the city sizes could be explained with Zipf law \cite{gabaix1999}. The number of citations scales as lognormal distribution \cite{radicchi2008}. In this thesis, we empirically analyzed the growth of online social systems. They consist of groups whose growth is universal. The empirical analyses of Meetup groups and Reddits showed their group size distribution follows universal lognormal distribution, stable over time. This research aimed to examine the structure and dynamics of the interaction network. We proposed the bipartite group model to gain a deeper understanding of the factors that affect the growth of social groups in a complex system. The growth in this model is driven by fluctuating signals, similar to the paper presented in chapter \ref{Ch:signals}: we use a time series of new members from Meetup and Reddit. The number of groups also grows as each user can create a new one; otherwise, the user joins the old group, and different linking rules determine his decision. One option is that the user joins a group where she already has friends; it's determined with affiliating probability  $p_{aff}$, while with probability $1-p_{aff}$, the user chooses a random group. Group size distribution in this model is lognormal. The width of the lognormal distribution depends on the probability $p_{aff}$; it becomes broader with a larger probability $p_{aff}$. 

%The lognormal distribution of the group sizes emerges when with probability $p_{aff}$, users prefer groups whose friends are already members, while with probability $1-p_{aff}$, their choice is random. The width of the lognormal distribution depends on the parameter $p_{aff}$. The systems influenced more by social connections have larger $p_{aff}$, and the broader group sizes have lognormal distribution. %TODO Ana paff nije objasnjeno ovde. Znam da je objašnjeno u Sekciji 4, ali sam zaključak treba da može da se razume i bez toga da se ulazi u detalje teze. Plus, ova osim recenice u sredini meni ove druge dve recenice nisu jasne, odnosno ne znam sta njima zelis da kazes.

In chapter \ref{Ch:Trust}, we focused on the factors that influence the sustainability of evolving complex networks. Specifically, we investigated the sustainability of social groups on the Question-Answers platform Stack Exchange. 
%on the Question-Answers platform Stack Exchange.
%%TODO we focused on the factors influencing the sustainability of evolving complex networks. Specifically, we investigated the sustainability of social groups on the Question-Answers platform Stack Exchange.
Each site goes through several phases before being successful and launched. During that period, the site may be closed. We selected several topics in which sites for the first time were closed, but in the second attempt, they survived and are still active. We provide a detailed analysis of active and closed Stack Exchange sites, compare their properties and identify what is crucial for the community's survival. We map user interactions observed in 30 days onto complex networks. Further, we slide the window by one day and follow the evolution of the network. 

According to the clustering properties of these networks, sustainable communities have a higher value of local cohesiveness. We use the Bayesian stochastic block modelling approach~\cite{gallagher2020clarified} to determine the core-periphery structure of these networks. We find that sustainable communities develop stable, better-connected cores. To analyze the evolution of collective trust in SE communities, we modify the Dynamic InteractionBased Reputation Model~\cite{melnikov2018toward} (DIBR) model. We use the DIBR model to measure the user's reputation based on the frequency of their activity and its evolution during the first 180 days. The trust between core members of active communities develops early and is higher than in closed communities during the first 180 days. The early emergence of a stable, trustworthy core may be a crucial factor in determining a knowledge-sharing community's sustainability. 

%%TODO: Pasus koji opisuje next directions 
The question raised by this study is how trust emerges among users in questions answers communities where the users tend to share knowledge, and their communication is neutral or positive. Some communities started promoting hate speech on different online platforms, resulting in the banning. But, banned users remained in the online world; they moved their communities to alternative platforms without strict policies, such as Voat. Later, Voat users also formed no-hate speech topics, and there is an open question does the emergence of trust differ among different communities? On the other hand, exploring higher-order representations of online communities would be interesting. Threads, where more people reply to one post, could be studied using simplicial complexes to reveal complex network structure patterns. Furthermore, the research that employs agent-based modelling allows us to connect closer the actions of single users with the emergence of collective phenomena and the rise and fall of trust in the system. 

The results from this thesis contribute to our knowledge about structure and dynamics of evolving complex networks and how they are mutually linked. We explored different factors that influence network growth, structural properties and sustainability. The growth signal impacts the network's structural properties, while social interactions affect group segmentation. The sustainability of evolving networks depends on core-periphery structure, the core's stability and users' ability to form a trustworthy core. Research presented in this thesis confirms that dynamics is linked with the structure of its interaction network, while the structure directly determines the function, organization and sustainability of complex systems.  

%%TODO Ovo je OK ako bi teza bila poverenjju. Mozes ostaviti ovaj paragraf, ali nesdotaju jos dve. 

%%TODO Prvi paragraf koji sumira sve tri stvari, nesto u skladu sa ovih par recenica:%
%%TODO %"In this thessis, we have explored different factors that influence the growth, structure and sustainability of complex networks. Our results show that growth signal critical influence on the structure of evolving networks, while their segmentation into social groups is influenced by social factors. The sustainability of evolving complex networks is influenced both by its core-periphery structure and stability of the core, as well as the ability of users to form cohesive trustworthy core. Research presented in this thesis confirms once more that dynamics of a complex system is inextricably linked with the structure its interaction network, Furthermore, our results show that the the structure of a complex network determines the function, organization and sustainability of a complex system."

%%TODO I treba nam dodatni paragraf sa otvorenim pitanjima. Ovo moram da razmislim malo, ali ako ti nesto padne na pamet, slobodno napisi.

Complex network theory is a rapidly growing field, but many open research questions exist. With the increase in the availability of the data of various complex systems, the analysis of complex networks becomes even more popular and shows excellent potential for future work. While we mostly understand how to describe the network's structure, and many methods are adapted to deal with evolving complex networks, we still need insights into how to design networks in order to control their properties, to prevent the epidemic outbreaks and to enhance or diminish information diffusion. Incorporating spatial or temporal constraints in network models could provide a more accurate picture of systems evolution. Community detection methods are beneficial for understanding network structure and function, it lacks methods that easily adapt to network changes over time. The current development of deep learning on graphs could fill existing gaps and provide more accurate predictions of complex network systems behavior.

\selectlanguage{serbianc}
\sffamily
\fontencoding{OT2}\fontfamily{Tempora-TLF}\selectfont

\selectlanguage{english}

	
%end{itemize}

