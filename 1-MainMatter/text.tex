\chapter{Introduction} % Main chapter title
%\selectlanguage{serbianc}
%\sffamily
%\fontencoding{OT2}\fontfamily{Tempora-TLF}\selectfont
%TODO: uvodni deo o prvom radu

 %Many real systems are composed of a large number of elements interacting with each other. Due to interactions, without any central force, the system exhibits the emergence of collective behaviour on the macro level\cite{kwapien2012}. Such a system is called a \textbf{Complex System} and its properties can not be predicted from the behaviour of the one individual \cite{ladyman2013}. An example of a complex system is the human brain. The structure of the brain network and its properties are fundamental for brain functioning, while an emergent phenomenon is a human intelligence. In societies, people's interactions lead to civilization, economy, formation of social groups. Also, the animal populations show different levels of organization that emerge from the individual's interactions \cite{boccaletti2006complex}. \cite{thurner2018} 

%Complex systems are all around us. They are seen in the ways that migrating birds organize themselves into flocking formations and that ants communicate to successfully forage. They are seen in the ways in which humans form social networks, and in the patterns of communication, social capital, and reputation that emerge from these networks. They are seen in the emergent power-law or fractal structures of plants, snowflakes, landslides, and galaxies, as well as in similar structural patterns of wealth and income distribution, stock market fluctuations, population distributions between cities, and patterns of urban development. Complex systems are often referred to as “wholes that are more than the sum of their parts,” wholes whose behaviour cannot be understood without looking at the individual components and how they interact.


%Waterloo Institute for Complexity Innovation (WICI) studies the formal aspects of complex systems, harnessing both quantitative and qualitative approaches. We investigate questions such as: What processes and structures define complex systems and characterize their outcomes? How can these be modelled and their implications understood? What real-world problems are best represented by complex systems, and what new insights are gained from a complex-systems lens? Most importantly, how can our understanding of complexity help us innovate better to address the world’s most intractable problems?



%Complex adaptive systems — predominantly living systems, including human social systems — exhibit all these features; but, in addition, they survive and reproduce within dynamic selection environments. To do so, they have sets of embedded rules that guide their action in response to their external environments. These rules evolve under selection pressure.








%TIme-series fractals (142 page)



%Networks

%Networks are a tool for keeping track of who is interacting with whom,
%at what strength, when, and in what way. Networks are essential for understanding
%co-evolution and phase diagrams of complex systems. Networks are also convenient for
%describing structures of objects, flows, and data. Practically everything that can be stored
%in a relational database is a network. Everything that can be related to, or associated with
%other things is part of a network. Mathematically, networks are matrices and, as such,
%are just a subset of linear algebra. Their importance and value for complex systems
%comes from their role in dynamical adaptive systems, where networks of interactions
%dynamically update the states of a system, and where the dynamics of states updates the
%interaction networks.

%Complex systems are typically characterized by multiple types of interaction that occur between specific types of elements. For
%instance, interactions in social networks may consist of two individuals that exchange
%information by meeting each other either physically or through certain communication
%networks, such as mobile phones or online social networks. People without access to
%certain communication channels consequently cannot interact through that mode of
%communication. Networks serve as a book-keeping system to specify who interacts with
%whom, in what way, how strongly, when, and for how long.

%The field of network science in its modern form was triggered by two contributions
%in the late 1990s [28, 402]. As called for by Erdős and Rényi, these works defined
%classes of random networks that replaced the hypothesis of equal linking probability
%for all connections by more realistic assumptions. Watts and Strogatz were interested
%in the so-called small world puzzle, which is based on two seemingly contradictory
%empirical observations in many social, biological, and technological networks. On the
%one hand, certain networks are known to display high clustering and to consist of densely
%interconnected groups of nodes [339]. One would thus expect such networks to require a
%large number of steps to be ‘traversed’, as when a walker is on the network, he will more
%often than not get stuck in local clusters of nodes. On the other hand, however, such
%networks can, in reality, often be traversed in just a few steps. In social networks it has
%been shown with famous ‘letter passing experiments’ that the average ‘distance’ between
%any two individuals on the globe is about six [375]; hence the popular concept of ‘six
%degrees of separation’ [105, 401]. Watts and Strogatz reconciled these two observations
%by understanding ‘small world networks’: they showed that, starting from a network with
%very high clustering and very high distances, it only takes the addition of a relatively
%small number of random links (shortcuts) to create the small world effect. This idea had
%been anticipated by Mark Granovetter long before; he conjectured that social networks
%maintain cohesiveness on large scales due to so-called weak ties that act as bridges
%between well-connected groups of people [156].

%The second work that started the boom of network science was contributed by
%Barabási and Albert [28]. Their point of departure was the observation that many real
%world networks, such as the world wide web, have probability distributions of node
%degrees that follow power laws [8]. This means that there exist nodes that are extremely
%well linked—hubs. They introduced a simple model for the growth of networks to
%account for this observation: the Barabási–Albert (BA) model [28]. Starting from a small
%network with only a couple of nodes, new nodes that are preferentially linked to well-
%connected nodes are continually added to the network.

%If Erdős and Rényi had already anticipated these developments in 1960, why did it
%take more than thirty years to realize them? Naturally, the answer has to do with data
%availability and computer power [188]. In no other field are these developments as
%evident as in biology, which has essentially become a science of molecular and cellular
%networks [29, 200, 230]. Following the long tradition of networks in social science [398],
%economics [110, 190, 331], finance [5, 138, 165], and public health [75, 80] are now
%also starting to embrace networks. Network analysis has become one of the standard
%tools in data analytics [73, 291], along with regression models, time series analysis, and
%generative statistical models. Real networks are often much more heterogeneous than
%random networks such as those obtained from the ER or BA models. Links may be
%directed and undirected, spatially and temporally embedded, may only exist between
%different types of nodes that, in turn, may be related through links of more than one type;
%and all these properties are subject to constant dynamical changes and to the birth and
%death processes of nodes and links [54, 232]. Interest is currently shifting from simple
%and often static networks to generalized network structures that consist of multiple types
%of links and nodes—again, just as Erdős and Rényi predicted sixty years ago.






