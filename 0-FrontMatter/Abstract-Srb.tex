\normalsize


\selectlanguage{serbianc}
\sffamily
\fontencoding{OT2}\fontfamily{Tempora-TLF}\selectfont

\chapter*{{\fontencoding{OT2}\fontfamily{Tempora-TLF}\selectfont Сажетак}}
\thispagestyle{empty} 

Комплексни системи се налазе свуда око нас у различитим доменима физике, биологије и друштвених наука. Иако се разликују по пореклу и функцији, заједничка карактеристика им је да се састоје од великог броја елемената који међусобно интерагују и због тих интеракција испоњавају колективно понашање. Комплексне мреже представљају општи приступ за репрезентацију образаца интеракција у комплексним системима. Структура комплексне мреже и њена еволуција су узајамно повезане са динамиком и функцијом комплексног система. Проналажење колективних фенома и разумевање како они настају из индивидуалних интеракција је један од важних истраживачких проблема. Теорија комплексних система нам пружа нове методе за истраживање комплексних система. Она комбинује методе статистичке физике, рачунарске физике, теорије комплексних мрежа, компјутерских наука како би квантитативно описала и проучавала различите колективне појаве и предложила теоријске моделе ради бољег разумевања механизама који су у основи динамике и еволуције комплексних мрежа. 

Ова теза има за циљ да прошири знање о структури и динамици растућих комплексних мрежа кроз анализу емпиријских података из различитих онлајн друштвених система и дефинисањем модела и теорија које би могле да објасне њихове специфичне карактеристике. Друштвени системи стално еволуирају и због тога је неопходно разумети везе између њихове структуре, раста и сегментације и како те везе утичу на њихову одрживост.

Ранији радови сугерисали су да својства сигнала раста утичу на структуру и динамику растућих комплексних мрежа. У реалним онлајн системима, сигнали раста флуктуирају током времена и они су дугодометно корелисани и имају мултифрактална својства. Као сигнале раста у овој тези, користимо временске серије нових корисника из реалних система {\selectlanguage{english}MySpace} и {\selectlanguage{english}TECH}, и компјутерски генерисане сигнале са специфичним својствима дугодометних корелација. Комбинујемо их са мрежним моделом старости чворова да бисмо детаљно испитали како карактеристике ових сигнала утичу на структуру комплесних мрежа. Наши резултати показују да својства сигнала раста имају најзначајнији утицај на структуру мрежа са широким степеном дистрибуције. За разлику од мрежа које имају константан раст, ове мреже су кластерисане и корелиране.

Даље, истражујемо како сигнал раста и правила повезивања утичу на сегментацију и раст социјалних група. Емпиријска анализа различитих друштвено-економских система указује на то да упркос разликама, ови системи често испољавају нека универзална својства у погледу сегментације и раста. Проучавали смо {\selectlanguage{english}Meetup} групе настале у Лондону и Њујорку, као и {\selectlanguage{english}subReddit} и открили да је дистрибуција величине група у овим системима логнормална и универзална током времена, локације и теме групе. Користили смо модел који комбинује два критеријума за повезивање корисника са друштвеним групама, насумично и на онову друштвених веза. Показали смо да су друштвене интеракције битан фактор при настанку логнормалне дистрибуције. Механизми под којима се корисници придружују друштвеним групама могу објаснити појаву универзалних својстава у друштвеном систему.

Комплексна теорија мрежа нам омогућава да опишемо како се развијају различита својства мреже и разумемо како еволуција утиче на њихову одрживост. Користили смо податке са {\selectlanguage{english}Stack Exchange} сајтова и упоређивали еволуцију структуре мреже за парове активних и затворених заједница током њихове ране фазе постојања. {\selectlanguage{english}Stack Exchange} сајтови су платформа за питања и одговоре на којима корисници деле знање о некој специфичној теми. Упоредили смо активне и затворене заједнице на четири теме, а то су астрономија, књижевност, економија и физика. Анализирали смо структурне обрасце у овим заједницама и открили да су активне више кластерисане и да их карактерише боље повезана и стабилност језгра. 
%Основни корисници су кључни за здрав сајт и морају да буду поуздани. 
Кроз динамички модел репутације измерили смо ниво поверења у овим заједницама. У активним заједницама, корисници који се налазе у језгру имају већу репутацију него у затвореним заједницама, што указује на важност да се стабилно језгро развије рано и да има висок ниво поверења.

\noindent
\textbf{Кључне речи:} статистичка физика комплексних система, структура и динамика комплексних мрежа, моделовање онлајн социјалних система \\
%{\textbf {Keywords:}} statistical physics of complex systems,  the structure and dynamics of complex networks, modeling online social systems \\%случајна секвенцијална адсорпција, хетерогени супстрати,
%парна корелациона функција\\
\textbf{Научна област:} Физика \\
\textbf{Ужа научна област:} Статистичка физика \\
\textbf{УДК број:} 536 %539.233, 536.12

\selectlanguage{english}

%\pagebreak
