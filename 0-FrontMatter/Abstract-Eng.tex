\selectlanguage{english}

\normalsize

\chapter{Abstract}


Network science provides an indispensable theoretical framework for
studying the structure and function of real complex systems. Different
network models are often used for finding the rules that govern their
evolution, whereby the correct choice of model details is crucial for
obtaining relevant insights. Here we study how the structure of
networks generated with the aging nodes model depends on the
properties of the growth signal. We use different fluctuating signals
and compare structural dissimilarities of the networks with those
obtained with a constant growth signal. We show that networks with
power-law degree distributions, which are obtained with time-varying
growth signals, are correlated and clustered, while networks obtained
with a constant growth signal are not. Indeed, the properties of the
growth signal significantly determine the topology of the obtained
networks and thus ought to be considered prominently in models of
complex systems.

Social groups are fundamental elements of any social system. Their emergence and evolution are closely related to the structure and dynamics of a social system. Research on social groups was primarily focused on the growth and the structure of the interaction networks of social system members and how members' group affiliation influences the evolution of these networks. The distribution of groups' size and how members join groups has not been investigated in detail. Here we combine statistical physics and complex network theory tools to analyze the distribution of group sizes in three data sets, Meetup groups based in London and New York and Reddit. We show that all three distributions exhibit log-normal behavior that indicates universal growth patterns in these systems. We propose a theoretical model that combines social and random diffusion of members between groups to simulate the roles of social interactions and members' interest in the growth of social groups. The simulation results show that our model reproduces growth patterns observed in empirical data. Moreover, our analysis shows that social interactions are more critical for the diffusion of members in online groups, such as Reddit, than in offline groups, such as Meetup. This work shows that social groups follow universal growth mechanisms that need to be considered in modeling the evolution of social systems. 

Knowledge-sharing communities are a fundamental element of any knowledge-based society. Understanding how they emerge, function, and disappear is thus of crucial importance. Many social and economic factors influence sustainable knowledge-sharing communities. Here we explore the role of the structure of social interactions and social trust in the emergence of these communities. Using tools from complex network theory, we analyze the early evolution of social structure in four pairs of StackExchange communities, each corresponding to one active and one closed community on the same topic. We adapt the dynamical reputation model to quantify the evolution of social trust in these communities. Our analysis shows that active communities have higher local cohesiveness and develop stable and more strongly connected cores. The average reputation is higher in sustainable communities. In these communities, the trust between core members develops early and remains high over time. Our results imply that efforts to create a stable and trustworthy core may be crucial for building a sustainable knowledge-sharing community.


\noindent
{\textbf {Keywords:}} random sequential adsorption, heterogeneous substrate, pair
correlation function\\
{\textbf {Research field:}} Physics \\
{\textbf {Research subfield:}} Statistical physics\\
\textbf{UDC number:} 539.233, 536.12

\hfill

\justify
