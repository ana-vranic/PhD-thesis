\selectlanguage{english}

\normalsize

\chapter{Abstract}

Complex systems are all around us and can be found in various domains of physics, biology, and social sciences. While they differ in origin and function, their common feature is that they consist of a large number of interacting units and that, due to these interactions, they exhibit collective behavior. Complex networks represent a general framework for representing interaction patterns in complex systems. The structure of a complex network and its evolution are inevitably linked to the dynamics and function of a complex system. Detecting the collective phenomena and understanding how they emerge from individual interactions is one of the important research problems. Complexity science gives us new ways to explore complex systems. Complexity science combines tools, methods, and paradigms of statistical and computational physics, complex network theory, and computer science to describe and study different collective phenomena quantitatively and propose theoretical models to better understand the mechanisms underlying dynamics and drive the evolution of complex networks.

This thesis aims to broaden the knowledge of the structure and dynamics of evolving complex networks by analyzing the empirical data from different online social systems and providing the models and theories that could explain their specific characteristics. Social systems constantly evolve, and because of that, it is necessary to understand the connections between their structure, growth, and segmentation and how these connections influence their sustainability.

Earlier works have suggested that the properties of growth signals influence the structure and dynamics of evolving complex networks. In real online systems, growth signals fluctuate over time, and they are long-range correlated and have multifractal properties. We use time series of new users from real systems, MySpace and TECH, and computer-generated signals with specific long-range correlation properties as growing signals. We combine them with a network model of aging nodes to examine in detail how the features of these signals shape the structure of complex networks. Our results show that the properties of the growth signal have the most substantial influence on the structure of networks with broad degree distribution. Unlike networks grown with constant signals, these networks are clustered and correlated.

Further, we explore the influence of growth signals and linking rules on the segmentation and growth of the social group in the social system. Empirical analysis of different socio-economic systems indicates that despite their differences, these systems often exhibit some universal properties regarding their segmentation and growth. We analyze the Meetup groups in London and New York and Subreddits and find that group size distribution in these systems is lognormal and universal over time, location, and topic. We use a model that interplays two criteria for users’ linking with social groups, random and based on social connections. We show that social interactions are an essential factor in the emergence of the lognormal distribution. We demonstrate that mechanisms under which users join social groups could explain the emergence of some universal properties in the social system.

The complex network theory allows us to determine how different network properties evolve and understand how this evolution influences their sustainability. We use data from Stack Exchange sites and compare the evolution of network structure for pairs of active and closed communities during their early phase of existence. Stack Exchange sites are question-and-answer platforms where users share knowledge on some specific topic. We compare active and closed communities on four topics, namely astronomy, literature, economics, and physics. We analyze the structural patterns in these communities and find that active ones are more clustered and characterized by better-connected and stable cores. Core users are crucial for a healthy site and need to be trustworthy. Through the dynamic reputation model, we measure the level of trust in these communities. In active communities, core users show a higher reputation than in closed communities, indicating the importance that a stable core develops early and has a high level of trust. \\~\\
{\textbf {Keywords:}} statistical physics of complex systems,  the structure and dynamics of complex networks, modeling online social systems \\
{\textbf {Research field:}} Physics \\
{\textbf {Research subfield:}} Statistical physics\\
\textbf{UDC number:} 536 %539.233, 536.12

\hfill
\justify
