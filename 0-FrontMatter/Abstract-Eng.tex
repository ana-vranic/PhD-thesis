\selectlanguage{english}

\normalsize

%\chapter{Evolving complex networks: structure and dynamics}

\chapter{Abstract}

Complexity science gives us new ways to explore complex systems. Detecting the collective phenomena and understanding how they emerge from individual interactions is one of the important research problems. Complex systems are all around us; they come from physical, biological, and social systems. The complex network is a general framework for representing interaction patterns in complex systems. The structure of the network could influence the behaviour of the system. The discovery that real-world networks are far from random and show scale-free properties and small-world phenomena led to the development of this field. New theories and models were needed to study socio-economics phenomena. At the same time, statistical and computational physics tools helped analyse and model their complex network representations. %The network models are essential for understanding the rules that govern complex networks' evolution.
This thesis aims to broaden the knowledge of complex networks by empirically analysing different online social systems and providing the models and theories that could explain their specific characteristics.  

The first part of the research explores how different growing signals influence the structure of complex networks. Over time, systems do not grow at a constant rate, and the networks that grow under fluctuating signals are clustered and correlated, while networks grown with a constant signal are not. Here, we systematically understand the connection between the growth signal and the network structure. For the growing network model, we use time series of new users from natural systems, such as MySpace and TECH. At the same time, computer-generated long-range correlated signals help distinguish which properties of time series shape the structure of complex networks. When signals are correlated and have multifractal properties, they mainly influence the scale-free networks promoting the creation of highly connected nodes. 

The second part of the research focuses on the evolution of large online platforms, where users organise into different kinds of social groups. These days, people interact intensively through online platforms. No matter whether the online systems rapidly grow, universal patterns in their growth stay stable. Our approach was to empirically analyse the evolution of three online systems: Meetup groups in London and New York and Subreddits. Their group size distributions follow log-normal, indicating the presence of universality.
On the other hand, it was important to identify the processes that led to the emergence of log-normal distribution and provide a model that could produce growth patterns in real systems. Social connections could be an important factor in the diffusion between groups. We used a model that interplays two criteria for group choices: random and based on social connections. We showed that social interactions are more critical in Subreddits than in Meetups for the diffusion between groups. 

The last part of the research, presented in the thesis, addresses what is necessary for one community to be sustainable. The complex network representation of the system allows us to determine how different network properties evolve. We use data from Stack Exchange sites, comparing communities on the same topic, but one was closed, and later when the site was proposed later, it stayed active until these days. Stack Exchange sites are question and answers platforms where users share knowledge. Analysing the structural patterns in these communities, we found active ones to be more clustered and characterised by better-connected cores. Core users are crucial for a healthy site and need to be trustworthy. Through the dynamic reputation model, we attempt to measure the level of trust in these communities. In active communities, core users show a higher reputation than in closed communities, indicating the importance that a stable core develops early and has a high level of trust. \\~\\
%\noindent
{\textbf {Keywords:}} \\ %random sequential adsorption, heterogeneous substrate, pair
%correlation function \\
{\textbf {Research field:}} Physics \\
{\textbf {Research subfield:}} Statistical physics\\
\textbf{UDC number:} 536 %539.233, 536.12

\hfill

\justify
