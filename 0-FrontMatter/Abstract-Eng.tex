\selectlanguage{english}

\normalsize

%\chapter{Evolving complex networks: structure and dynamics}

\chapter{Abstract}
Complex systems are all around us and can be found in various domains of physics, biology, and social sciences. 
%Complex systems surround us, including physical, biological, and social systems. 
While they differ in origin and function, their common feature is that they consist of a large number of interacting units and that, due to these interactions, they exhibit collective behavior. Complex networks represent a general framework for representing interaction patterns in complex systems. The structure of a complex network and its evolution are inevitably linked to the dynamics and function of a complex system. Detecting the collective phenomena and understanding how they emerge from individual interactions is one of the important research problems. Complexity science gives us new ways to explore complex systems. Complexity science combines tools, methods, and paradigms of statistical and computational physics, complex network theory, and computer science to describe and study different collective phenomena quantitatively and propose theoretical models to better understand the mechanisms underlying dynamics and drive the evolution of complex networks.

This thesis aims to broaden the knowledge of the structure and dynamics of evolving complex networks by analyzing the empirical data from different online social systems and providing the models and theories that could explain their specific characteristics. Social systems constantly evolve, and because of that, it is necessary to understand the connections between their structure, growth, and segmentation and how these connections influence their sustainability.

Earlier works have suggested that the properties of growth signals influence the structure and dynamics of evolving complex networks. In real online systems, growth signals fluctuate over time, and they are long-range correlated and have multifractal properties. We use time series of new users from real systems, MySpace and TECH, and computer-generated signals with specific long-range correlation properties as growing signals. We combine them with a network model of aging nodes to examine in detail how the features of these signals shape the structure of complex networks. Our results show that the properties of the growth signal have the most substantial influence on the structure of networks with broad degree distribution. Unlike networks grown with constant signals, these networks are clustered and correlated.

Further, we explore the influence of growth signals and linking rules on the segmentation and growth of the social group in the social system. Empirical analysis of different socio-economic systems indicates that despite their differences, these systems often exhibit some universal properties regarding their segmentation and growth. We analyze the Meetup groups in London and New York and Subreddits and find that group size distribution in these systems is lognormal and universal over time, location, and topic. We use a model that interplays two criteria for users’ linking with social groups, random and based on social connections. We show that social interactions are an essential factor in the emergence of the lognormal distribution. We demonstrate that mechanisms under which users join social groups could explain the emergence of some universal properties in the social system.

The complex network theory allows us to determine how different network properties evolve and
understand how this evolution influences their sustainability. We use data from Stack Exchange sites and compare the evolution of network structure for pairs of active and closed communities during their early phase of existence. Stack Exchange sites are question-and-answer platforms where users share knowledge on some specific topic. We compare active and closed communities on four topics, namely astronomy, literature, economics, and physics. We analyze the structural patterns in these communities and find that active ones are more clustered and characterized by better-connected and stable cores. Core users are crucial for a healthy site and need to be trustworthy. Through the dynamic reputation model, we measure the level of trust in these communities. In active communities, core users show a higher reputation than in closed communities, indicating the importance that a stable core develops early and has a high level of trust. \\~\\
{\textbf {Keywords:}} statistical physics of complex systems,  the structure and dynamics of complex networks, modeling online social systems \\ %random sequential adsorption, heterogeneous substrate, pair
%correlation function \\
{\textbf {Research field:}} Physics \\
{\textbf {Research subfield:}} Statistical physics\\
\textbf{UDC number:} 536 %539.233, 536.12

%Complexity science gives us new ways to explore complex systems. Detecting the collective phenomena and understanding how they emerge from individual interactions is one of the important research problems. Complex systems are all around us; they come from physical, biological, and social systems. The complex network is a general framework for representing interaction patterns in complex systems. The structure of the network could influence the behavior of the system. The discovery that real-world networks are far from random and show scale-free properties and small-world phenomena led to the development of this field. New theories and models were needed to study socio-economics phenomena. At the same time, statistical and computational physics tools helped analyze and model their complex network representations. This thesis aims to broaden the knowledge of complex networks by analyzing different online social systems and providing the models and theories that could explain their specific characteristics. 

%Social systems constantly evolve, and because of that, it is necessary to understand the connections between the growth of the network and the network structure. In real online systems, growth signals fluctuate over time; they are long-range correlated and have multifractal properties. As a consequence, networks become clustered and also correlated. We use time series of new users from real systems, such as MySpace and  TECH, for growing signals. At the same time, computer-generated signals with specific long-range correlated properties help us to identify features that shape the structure of complex networks. When signals are correlated and have multifractal properties, they mainly influence the scale-free networks promoting the creation of highly connected nodes.

%On the other hand, the mechanisms under which users join social groups could explain the emergence of some universal properties in the system. These days people interact intensively through online platforms, and no matter whether the online systems rapidly grow, universal patterns in their growth stay stable. Analyzing the Meetup groups in London and New York and Subreddits, we found that group size distribution in these systems is lognormal and universal over time. We used a model that interplays two criteria for users' group choices: random and based on social connections. We showed that social interactions are an important factor in the emergence of the lognormal distribution.

%The complex network theory allows us to determine how different network properties evolve and understand how communities become sustainable. We use data from Stack Exchange sites, comparing communities on the same topic, but one is closed and the other active until these days. Stack Exchange sites are question and answers platforms where users share knowledge. Analyzing the structural patterns in these communities, we found active ones to be more clustered and characterized by better-connected cores. Core users are crucial for a healthy site and need to be trustworthy. Through the dynamic reputation model, we attempt to measure the level of trust in these communities. In active communities, core users show a higher reputation than in closed communities, indicating the importance that a stable core develops early and has a high level of trust. \\~\\



%The first part of the research explores how different growing signals influence the structure of complex networks. Over time, systems do not grow at a constant rate, and the networks that grow under fluctuating signals are clustered and correlated, while networks grown with a constant signal are not. Here, we systematically understand the connection between the growth signal and the network structure. For the growing network model, we use time series of new users from natural systems, such as MySpace and TECH. At the same time, computer-generated long-range correlated signals help distinguish which properties of time series shape the structure of complex networks. When signals are correlated and have multifractal properties, they mainly influence the scale-free networks promoting the creation of highly connected nodes. 

%The second part of the research focuses on the evolution of large online platforms, where users organise into different kinds of social groups. These days, people interact intensively through online platforms. No matter whether the online systems rapidly grow, universal patterns in their growth stay stable. Our approach was to empirically analyse the evolution of three online systems: Meetup groups in London and New York and Subreddits. Their group size distributions follow log-normal, indicating the presence of universality. 
%On the other hand, it was important to identify the processes that led to the emergence of log-normal distribution and provide a model that could produce growth patterns in real systems. Social connections could be an important factor in the diffusion between groups. We used a model that interplays two criteria for group choices: random and based on social connections. We showed that social interactions are more critical in Subreddits than in Meetups for the diffusion between groups. 

%The last part of the research, presented in the thesis, addresses what is necessary for one community to be sustainable. The complex network representation of the system allows us to determine how different network properties evolve. We use data from Stack Exchange sites, comparing communities on the same topic, but one was closed, and later when the site was proposed later, it stayed active until these days. Stack Exchange sites are question and answers platforms where users share knowledge. Analysing the structural patterns in these communities, we found active ones to be more clustered and characterised by better-connected cores. Core users are crucial for a healthy site and need to be trustworthy. Through the dynamic reputation model, we attempt to measure the level of trust in these communities. In active communities, core users show a higher reputation than in closed communities, indicating the importance that a stable core develops early and has a high level of trust. \\~\\
%\noindent


\hfill

\justify
